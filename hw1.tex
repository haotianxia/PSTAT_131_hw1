% Options for packages loaded elsewhere
\PassOptionsToPackage{unicode}{hyperref}
\PassOptionsToPackage{hyphens}{url}
%
\documentclass[
]{article}
\usepackage{amsmath,amssymb}
\usepackage{lmodern}
\usepackage{iftex}
\ifPDFTeX
  \usepackage[T1]{fontenc}
  \usepackage[utf8]{inputenc}
  \usepackage{textcomp} % provide euro and other symbols
\else % if luatex or xetex
  \usepackage{unicode-math}
  \defaultfontfeatures{Scale=MatchLowercase}
  \defaultfontfeatures[\rmfamily]{Ligatures=TeX,Scale=1}
\fi
% Use upquote if available, for straight quotes in verbatim environments
\IfFileExists{upquote.sty}{\usepackage{upquote}}{}
\IfFileExists{microtype.sty}{% use microtype if available
  \usepackage[]{microtype}
  \UseMicrotypeSet[protrusion]{basicmath} % disable protrusion for tt fonts
}{}
\makeatletter
\@ifundefined{KOMAClassName}{% if non-KOMA class
  \IfFileExists{parskip.sty}{%
    \usepackage{parskip}
  }{% else
    \setlength{\parindent}{0pt}
    \setlength{\parskip}{6pt plus 2pt minus 1pt}}
}{% if KOMA class
  \KOMAoptions{parskip=half}}
\makeatother
\usepackage{xcolor}
\IfFileExists{xurl.sty}{\usepackage{xurl}}{} % add URL line breaks if available
\IfFileExists{bookmark.sty}{\usepackage{bookmark}}{\usepackage{hyperref}}
\hypersetup{
  pdftitle={Homework 1},
  pdfauthor={Haotian Xia(5069182)},
  hidelinks,
  pdfcreator={LaTeX via pandoc}}
\urlstyle{same} % disable monospaced font for URLs
\usepackage[margin=1in]{geometry}
\usepackage{color}
\usepackage{fancyvrb}
\newcommand{\VerbBar}{|}
\newcommand{\VERB}{\Verb[commandchars=\\\{\}]}
\DefineVerbatimEnvironment{Highlighting}{Verbatim}{commandchars=\\\{\}}
% Add ',fontsize=\small' for more characters per line
\usepackage{framed}
\definecolor{shadecolor}{RGB}{248,248,248}
\newenvironment{Shaded}{\begin{snugshade}}{\end{snugshade}}
\newcommand{\AlertTok}[1]{\textcolor[rgb]{0.94,0.16,0.16}{#1}}
\newcommand{\AnnotationTok}[1]{\textcolor[rgb]{0.56,0.35,0.01}{\textbf{\textit{#1}}}}
\newcommand{\AttributeTok}[1]{\textcolor[rgb]{0.77,0.63,0.00}{#1}}
\newcommand{\BaseNTok}[1]{\textcolor[rgb]{0.00,0.00,0.81}{#1}}
\newcommand{\BuiltInTok}[1]{#1}
\newcommand{\CharTok}[1]{\textcolor[rgb]{0.31,0.60,0.02}{#1}}
\newcommand{\CommentTok}[1]{\textcolor[rgb]{0.56,0.35,0.01}{\textit{#1}}}
\newcommand{\CommentVarTok}[1]{\textcolor[rgb]{0.56,0.35,0.01}{\textbf{\textit{#1}}}}
\newcommand{\ConstantTok}[1]{\textcolor[rgb]{0.00,0.00,0.00}{#1}}
\newcommand{\ControlFlowTok}[1]{\textcolor[rgb]{0.13,0.29,0.53}{\textbf{#1}}}
\newcommand{\DataTypeTok}[1]{\textcolor[rgb]{0.13,0.29,0.53}{#1}}
\newcommand{\DecValTok}[1]{\textcolor[rgb]{0.00,0.00,0.81}{#1}}
\newcommand{\DocumentationTok}[1]{\textcolor[rgb]{0.56,0.35,0.01}{\textbf{\textit{#1}}}}
\newcommand{\ErrorTok}[1]{\textcolor[rgb]{0.64,0.00,0.00}{\textbf{#1}}}
\newcommand{\ExtensionTok}[1]{#1}
\newcommand{\FloatTok}[1]{\textcolor[rgb]{0.00,0.00,0.81}{#1}}
\newcommand{\FunctionTok}[1]{\textcolor[rgb]{0.00,0.00,0.00}{#1}}
\newcommand{\ImportTok}[1]{#1}
\newcommand{\InformationTok}[1]{\textcolor[rgb]{0.56,0.35,0.01}{\textbf{\textit{#1}}}}
\newcommand{\KeywordTok}[1]{\textcolor[rgb]{0.13,0.29,0.53}{\textbf{#1}}}
\newcommand{\NormalTok}[1]{#1}
\newcommand{\OperatorTok}[1]{\textcolor[rgb]{0.81,0.36,0.00}{\textbf{#1}}}
\newcommand{\OtherTok}[1]{\textcolor[rgb]{0.56,0.35,0.01}{#1}}
\newcommand{\PreprocessorTok}[1]{\textcolor[rgb]{0.56,0.35,0.01}{\textit{#1}}}
\newcommand{\RegionMarkerTok}[1]{#1}
\newcommand{\SpecialCharTok}[1]{\textcolor[rgb]{0.00,0.00,0.00}{#1}}
\newcommand{\SpecialStringTok}[1]{\textcolor[rgb]{0.31,0.60,0.02}{#1}}
\newcommand{\StringTok}[1]{\textcolor[rgb]{0.31,0.60,0.02}{#1}}
\newcommand{\VariableTok}[1]{\textcolor[rgb]{0.00,0.00,0.00}{#1}}
\newcommand{\VerbatimStringTok}[1]{\textcolor[rgb]{0.31,0.60,0.02}{#1}}
\newcommand{\WarningTok}[1]{\textcolor[rgb]{0.56,0.35,0.01}{\textbf{\textit{#1}}}}
\usepackage{graphicx}
\makeatletter
\def\maxwidth{\ifdim\Gin@nat@width>\linewidth\linewidth\else\Gin@nat@width\fi}
\def\maxheight{\ifdim\Gin@nat@height>\textheight\textheight\else\Gin@nat@height\fi}
\makeatother
% Scale images if necessary, so that they will not overflow the page
% margins by default, and it is still possible to overwrite the defaults
% using explicit options in \includegraphics[width, height, ...]{}
\setkeys{Gin}{width=\maxwidth,height=\maxheight,keepaspectratio}
% Set default figure placement to htbp
\makeatletter
\def\fps@figure{htbp}
\makeatother
\setlength{\emergencystretch}{3em} % prevent overfull lines
\providecommand{\tightlist}{%
  \setlength{\itemsep}{0pt}\setlength{\parskip}{0pt}}
\setcounter{secnumdepth}{-\maxdimen} % remove section numbering
\ifLuaTeX
  \usepackage{selnolig}  % disable illegal ligatures
\fi

\title{Homework 1}
\author{Haotian Xia(5069182)}
\date{April 03, 2022}

\begin{document}
\maketitle

Question 1: Supervised learning prediction is accurately predict future
response given predictors. Unsupervised learning is no response
variables. For Supervised learning, the responses are known and examples
must be labeled. For Unsupervised learning, the responses are not known
and examples are not labeled .

Question2: Classification is the process of finding or discovering a
model or function which helps in separating the data into multiple
categorical classes. The response variable is an categorical
varible(discrete values). Regression is the process of finding a model
or function for distinguishing the data into continuous real values
instead of using classes or discrete values. Again, Regression aims to
predict a continuous output value, and Classification aims to predict
which class (a discrete integer or categorical label) the input
corresponds to.

Question3: Regression model metrics: Mean Squared Error (MSE). Root Mean
Squared Error (RMSE). Classification model metrics: Confusion Matrix \&
Precision, Recall, and F-1 Score, ROC AUC.

Question4: DESCRIPTIVE: basically choose model to best visually
emphasize a trend in data. Inferential models: it aims to find what
features are significant.Aim is to test theories (Possibly) causal
claims. State relationship between outcome \& predictor(s) Predictive
models: It chooses the best combo of features that fit best. Aim is to
predict Y(response variable) with minimum reducible error. Not focused
on hypothesis tests. (From lecture note)

Question5: A mechanistic model uses a theory to predict what will happen
in the real world. It predicts future based on a theory.
Empirically-driven studies real-world events to develop a theory. The
difference is mechanistic model Assume a parametric form for
\(f(i.e.\beta_0+\beta_1+⋯)\) and it won't match true unknown
f.~Empirically-driven does not have assumptions about f.~mechanistic
model need to add more parameters to have more flexibility.
Empirically-driven model have much more flexible by default. The similar
thing is both of them will overfitting. (3) For the empirically-driven
model, it is easier than mechanistic model in overfitting(Small train
MSE, large test MSE, low bias and high variance) becuase
empirically-driven model are more flexible while learning a target
function. So when we want to do the empirically-driven model, we need to
make sure we are not overfitting which means we do not train too complex
f function. When we do mechanistic model, since we already assumed f
function, it will either overfitting or underfitting(high bias, and low
variance), so we need to add more variable or change our function
assumption to find the best trade off between bias and variance.

Question6: the first part is predictive because we are trying to predict
who they will vote based on their profile(predictor variables). we need
an actual predicted result whether who will win the election. the second
question is inferential because we are trying to test if the specific
predictor variables(had personal contact with the candidate) is
significant or not.

\begin{Shaded}
\begin{Highlighting}[]
\FunctionTok{library}\NormalTok{(tidyverse)}
\end{Highlighting}
\end{Shaded}

\begin{verbatim}
## -- Attaching packages --------------------------------------- tidyverse 1.3.1 --
\end{verbatim}

\begin{verbatim}
## v ggplot2 3.3.5     v purrr   0.3.4
## v tibble  3.1.6     v dplyr   1.0.8
## v tidyr   1.2.0     v stringr 1.4.0
## v readr   2.1.2     v forcats 0.5.1
\end{verbatim}

\begin{verbatim}
## -- Conflicts ------------------------------------------ tidyverse_conflicts() --
## x dplyr::filter() masks stats::filter()
## x dplyr::lag()    masks stats::lag()
\end{verbatim}

\begin{Shaded}
\begin{Highlighting}[]
\FunctionTok{library}\NormalTok{(tidymodels)}
\end{Highlighting}
\end{Shaded}

\begin{verbatim}
## -- Attaching packages -------------------------------------- tidymodels 0.2.0 --
\end{verbatim}

\begin{verbatim}
## v broom        0.7.12     v rsample      0.1.1 
## v dials        0.1.0      v tune         0.2.0 
## v infer        1.0.0      v workflows    0.2.6 
## v modeldata    0.1.1      v workflowsets 0.2.1 
## v parsnip      0.2.1      v yardstick    0.0.9 
## v recipes      0.2.0
\end{verbatim}

\begin{verbatim}
## -- Conflicts ----------------------------------------- tidymodels_conflicts() --
## x scales::discard() masks purrr::discard()
## x dplyr::filter()   masks stats::filter()
## x recipes::fixed()  masks stringr::fixed()
## x dplyr::lag()      masks stats::lag()
## x yardstick::spec() masks readr::spec()
## x recipes::step()   masks stats::step()
## * Learn how to get started at https://www.tidymodels.org/start/
\end{verbatim}

\begin{Shaded}
\begin{Highlighting}[]
\FunctionTok{library}\NormalTok{(ISLR)}
\FunctionTok{library}\NormalTok{(ggplot2)}
\end{Highlighting}
\end{Shaded}

\begin{Shaded}
\begin{Highlighting}[]
\NormalTok{data}\OtherTok{\textless{}{-}}\NormalTok{ mpg}
\FunctionTok{head}\NormalTok{(data)}
\end{Highlighting}
\end{Shaded}

\begin{verbatim}
## # A tibble: 6 x 11
##   manufacturer model displ  year   cyl trans      drv     cty   hwy fl    class 
##   <chr>        <chr> <dbl> <int> <int> <chr>      <chr> <int> <int> <chr> <chr> 
## 1 audi         a4      1.8  1999     4 auto(l5)   f        18    29 p     compa~
## 2 audi         a4      1.8  1999     4 manual(m5) f        21    29 p     compa~
## 3 audi         a4      2    2008     4 manual(m6) f        20    31 p     compa~
## 4 audi         a4      2    2008     4 auto(av)   f        21    30 p     compa~
## 5 audi         a4      2.8  1999     6 auto(l5)   f        16    26 p     compa~
## 6 audi         a4      2.8  1999     6 manual(m5) f        18    26 p     compa~
\end{verbatim}

EX1: we notice that the mpg 26 has the largest number of car among the
dataset. following by mpg 17 and 29. We can deep dive into those cars
that have mpg 26 or 17 or 29 to see if there are some similarity(such as
years, class etc.) in each mpg group. By looking at the distribution,
most cars can drive 15\textasciitilde29 miles per gallon.

\begin{Shaded}
\begin{Highlighting}[]
\FunctionTok{ggplot}\NormalTok{(data, }\FunctionTok{aes}\NormalTok{(}\AttributeTok{x=}\NormalTok{hwy)) }\SpecialCharTok{+} \FunctionTok{geom\_histogram}\NormalTok{(}\AttributeTok{binwidth=}\DecValTok{1}\NormalTok{)}
\end{Highlighting}
\end{Shaded}

\includegraphics{hw1_files/figure-latex/ex1-1.pdf} EX2:we notice that
cars have higher hwy have higher cty. By the blue trend line, there is
an almost positive linear relationship.

\begin{Shaded}
\begin{Highlighting}[]
\FunctionTok{ggplot}\NormalTok{(data, }\FunctionTok{aes}\NormalTok{(}\AttributeTok{x=}\NormalTok{hwy, }\AttributeTok{y=}\NormalTok{cty)) }\SpecialCharTok{+} \FunctionTok{geom\_point}\NormalTok{() }\SpecialCharTok{+} \FunctionTok{geom\_smooth}\NormalTok{()}
\end{Highlighting}
\end{Shaded}

\begin{verbatim}
## `geom_smooth()` using method = 'loess' and formula 'y ~ x'
\end{verbatim}

\includegraphics{hw1_files/figure-latex/ex2-1.pdf}

EX3: dodge produced the most cars, Lincoin produced the least.

\begin{Shaded}
\begin{Highlighting}[]
\FunctionTok{library}\NormalTok{(dplyr)}
\FunctionTok{ggplot}\NormalTok{(mpg, }\FunctionTok{aes}\NormalTok{(}\AttributeTok{x=}\FunctionTok{reorder}\NormalTok{(manufacturer,manufacturer,}\ControlFlowTok{function}\NormalTok{(x)}\SpecialCharTok{+}\FunctionTok{length}\NormalTok{(x))))}\SpecialCharTok{+}
  \FunctionTok{geom\_bar}\NormalTok{(}\AttributeTok{fill=}\StringTok{"steelblue"}\NormalTok{) }\SpecialCharTok{+} \FunctionTok{coord\_flip}\NormalTok{()}
\end{Highlighting}
\end{Shaded}

\includegraphics{hw1_files/figure-latex/ex3-1.pdf} EX4: We notice that
cars have more number of cylinders(cyl) will have lower highway miles
per gallon(hwy). In other words, with one gallon, less number of
cylinders cars can be drived more miles than larger number of cylinders
cars.

\begin{Shaded}
\begin{Highlighting}[]
\FunctionTok{ggplot}\NormalTok{(data, }\FunctionTok{aes}\NormalTok{(}\AttributeTok{x=}\NormalTok{ cyl,}\AttributeTok{y=}\NormalTok{hwy,}\AttributeTok{group =}\NormalTok{ cyl)) }\SpecialCharTok{+} 
  \FunctionTok{geom\_boxplot}\NormalTok{()}
\end{Highlighting}
\end{Shaded}

\includegraphics{hw1_files/figure-latex/ex4-1.pdf} EX5: This chart shows
that both cyl\&displ and hwy\&cty have positive correlations. These two
positive correlations make sense to me. For cyl\&displ, according to the
website, cars that have more cylinders suppose to have higher engine
displacement. For hwy\&cty, both variable measures how many miles a car
can drive(driving range). The only difference is one on the highway, and
the other in the city. they should have similar attributes and results.
cty\&displ, cty\&cyl,hwy\&displ, and hwy\&cyl are all have negative
correlations. These also make sense to me. According to the website,
more engine displacement cars have less driving range(more displ, less
hwy\&cty). Also, more cylinders cars have less driving range.

\begin{Shaded}
\begin{Highlighting}[]
\FunctionTok{library}\NormalTok{(corrplot)}
\end{Highlighting}
\end{Shaded}

\begin{verbatim}
## corrplot 0.92 loaded
\end{verbatim}

\begin{Shaded}
\begin{Highlighting}[]
\NormalTok{M }\OtherTok{=}\NormalTok{ dplyr}\SpecialCharTok{::}\FunctionTok{select\_if}\NormalTok{(data,is.numeric) }
\NormalTok{df }\OtherTok{=} \FunctionTok{subset}\NormalTok{(M, }\AttributeTok{select=}\FunctionTok{c}\NormalTok{(}\StringTok{\textquotesingle{}displ\textquotesingle{}}\NormalTok{,}\StringTok{\textquotesingle{}cyl\textquotesingle{}}\NormalTok{,}\StringTok{\textquotesingle{}cty\textquotesingle{}}\NormalTok{,}\StringTok{\textquotesingle{}hwy\textquotesingle{}}\NormalTok{))}
\FunctionTok{corrplot}\NormalTok{(}\FunctionTok{cor}\NormalTok{(df),}\AttributeTok{method =} \StringTok{\textquotesingle{}number\textquotesingle{}}\NormalTok{, }\AttributeTok{order =} \StringTok{\textquotesingle{}AOE\textquotesingle{}}\NormalTok{, }\AttributeTok{type =} \StringTok{\textquotesingle{}lower\textquotesingle{}}\NormalTok{, }\AttributeTok{diag =} \ConstantTok{FALSE}\NormalTok{)}
\end{Highlighting}
\end{Shaded}

\includegraphics{hw1_files/figure-latex/ex5-1.pdf} EX6:

\begin{Shaded}
\begin{Highlighting}[]
\FunctionTok{library}\NormalTok{(ggthemes)}

\CommentTok{\#p \textless{}{-} ggplot(data= mpg, aes(x=class, y =hwy)) + geom\_jitter(alpha = 0.5,height = 0)+ geom\_boxplot(alpha = 0.2) + coord\_flip() + theme\_gdocs()}
\NormalTok{mpg }\SpecialCharTok{\%\textgreater{}\%}
 \FunctionTok{ggplot}\NormalTok{() }\SpecialCharTok{+} \FunctionTok{geom\_boxplot}\NormalTok{(}\AttributeTok{mapping =} \FunctionTok{aes}\NormalTok{(}\AttributeTok{x =}\NormalTok{ class, }\AttributeTok{y =}\NormalTok{ hwy)) }\SpecialCharTok{+} \FunctionTok{geom\_jitter}\NormalTok{(}\AttributeTok{mapping =} \FunctionTok{aes}\NormalTok{(}\AttributeTok{x =}\NormalTok{ class, }\AttributeTok{y =}\NormalTok{ hwy), }\AttributeTok{alpha =} \FloatTok{0.25}\NormalTok{, }\AttributeTok{height =} \DecValTok{0}\NormalTok{) }\SpecialCharTok{+} \FunctionTok{theme\_gdocs}\NormalTok{() }\SpecialCharTok{+} 
  \FunctionTok{labs}\NormalTok{(}\AttributeTok{x =} \StringTok{"Vehicle Class"}\NormalTok{, }\AttributeTok{y =} \StringTok{"Highway MPG"}\NormalTok{)}\SpecialCharTok{+}
  \FunctionTok{coord\_flip}\NormalTok{()}
\end{Highlighting}
\end{Shaded}

\includegraphics{hw1_files/figure-latex/ex6-1.pdf} EX7:

\begin{Shaded}
\begin{Highlighting}[]
\NormalTok{mpg }\SpecialCharTok{\%\textgreater{}\%} 
  \FunctionTok{ggplot}\NormalTok{(}\FunctionTok{aes}\NormalTok{(}\AttributeTok{x=}\NormalTok{class, }\AttributeTok{y=}\NormalTok{hwy, }\AttributeTok{fill=}\FunctionTok{factor}\NormalTok{(drv))) }\SpecialCharTok{+}
  \FunctionTok{geom\_boxplot}\NormalTok{() }
\end{Highlighting}
\end{Shaded}

\includegraphics{hw1_files/figure-latex/ex7-1.pdf} EX8:

\begin{Shaded}
\begin{Highlighting}[]
\FunctionTok{ggplot}\NormalTok{(mpg,}\FunctionTok{aes}\NormalTok{(}\AttributeTok{x =}\NormalTok{ displ, }\AttributeTok{y =}\NormalTok{ hwy,}\AttributeTok{fill=}\NormalTok{drv,}\AttributeTok{color =}\NormalTok{drv)) }\SpecialCharTok{+}
  \FunctionTok{geom\_point}\NormalTok{()}\SpecialCharTok{+}
  \FunctionTok{geom\_smooth}\NormalTok{(}\FunctionTok{aes}\NormalTok{(}\AttributeTok{group=}\NormalTok{drv,}\AttributeTok{linetype =}\NormalTok{ drv), }\AttributeTok{color =} \StringTok{"blue"}\NormalTok{,}\AttributeTok{se=}\ConstantTok{FALSE}\NormalTok{,)}
\end{Highlighting}
\end{Shaded}

\begin{verbatim}
## `geom_smooth()` using method = 'loess' and formula 'y ~ x'
\end{verbatim}

\includegraphics{hw1_files/figure-latex/ex8-1.pdf}

\end{document}
